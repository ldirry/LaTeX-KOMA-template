%%%% Time-stamp: <2018-03-24 14:05:09 vk>
%% ========================================================================
%%%% Disclaimer
%% ========================================================================
%%
%% created by
%%
%%      Karl Voit
%%

%% ========================================================================
%%%% Basic settings
%% ========================================================================
%% (idea of using newcommands for basic documentclass settings from: Thomas Schlager)

\newcommand{\mypapersize}{A4}
%% e.g., "A4", "letter", "legal", "executive", ...
%% The size of the paper of the resulting PDF file.

\newcommand{\mylaterality}{twoside}
%% "oneside" or "twoside"
%% Either you are creating a document which is printed on both, left pages
%% and right pages (twoside) or you create a document which is printed
%% on right pages only (oneside).

\newcommand{\mydraft}{false}
%% "true" or "false"
%% Use draft mode? If true, included graphics are replaced by empty
%% rectangles (of same size) and overfull boxes (in margin space) are
%% marked with black box (-> easy to spot!)

\newcommand{\myparskip}{half}
%% e.g., "no", "full", "half", ...
%% How to separate paragraphs: indention ("no") or spacing ("half",
%% "full", ...).

\newcommand{\myBCOR}{0mm}
%% Inner binding correction. This value depends on the method which is
%% being used to bind your printed result. Some techniques do not
%% require a binding correction at all ("0mm"), other require for
%% example "5mm". Refer to KOMA script documentation for a detailed
%% explanation what a binding correction is and how to measure it.

\newcommand{\myfontsize}{12pt}
%% e.g., 10pt, 11pt, 12pt
%% The font size of the main text in pt (points).

\newcommand{\mylinespread}{1.0}
%% e.g., 1.0, 1.5, 2.0
%% Line spacing in %/100. For example 1.5 means 150% of the usual line
%% spacing. Please use with caution: 100% ("1.0") is fine because the
%% font was designed for it.

\newcommand{\mylanguage}{ngerman,american}
%% "english,ngerman", "ngerman,english", ...
%% NOTE: The *last* language is the active one!
%% See babel documentation for further details.

%% BibLaTeX-settings: (see biblatex reference for further description)
\newcommand{\mybiblatexstyle}{authoryear}
%% e.g., "alphabetic", "authoryear", ...
%% The biblatex style which is being used for referencing. See
%% biblatex documentation for further details and more values.
%%
%% CAUTION: if you change the style, please check for (in)compatible
%%          "biblatex" package options in the file
%%          "template/preamble.tex"! For example: "alphabetic" does
%%          not have an option "dashed=..." and causes an error if it
%%          does not get removed from the list of options.

\newcommand{\mybiblatexdashed}{false}  %% "true" or "false"
%% If true: replace recurring reference authors with a dash.

\newcommand{\mybiblatexbackref}{true}  %% "true" or "false"
%% If true: create backward links from reference to citations.

\newcommand{\mybiblatexfile}{references-biblatex.bib}
%% Name of the biblatex file that holds the references.

\newcommand{\mydispositioncolor}{30,103,182}
%% e.g., "30,103,182" (blue/turquois), "0,0,0" (black), ...
%% Color of the headings and so forth in RGB (red,green,blue) values.
%% NOTE: if you are using "0,0,0" for black, printers might still
%%       recognize pages as color pages. In case this is a problem
%%       (paying for color print-outs vs. paying for b/w-printouts)
%%       please edit file "template/preamble.tex" and change
%%       "\definecolor{DispositionColor}{RGB}{\mydispositioncolor}"
%%       to "\definecolor{DispositionColor}{gray}{0}" and thus
%%       overwriting the value of \mydispositioncolor above.

\newcommand{\mycolorlinks}{true}  %% "true" or "false"
%% Enables or disables colored links (hyperref package).

\newcommand{\mytitlepage}{template/title_Thesis_TU_Graz}
%% Your own or one of following pre-defined title pages:
%% "template/title_plain_maketitle": simple maketitle page
%% "template/title_Diplomarbeit_KF_Uni_Graz.tex": fancy (german) title page for KF Uni Graz
%% "template/title_Thesis_TU_Graz":
%%             titlepage for Graz University of Technology (correct
%%             (old?) Corporate Design) by Karl Voit (2012)
%% "template/title_Thesis_TU_Graz_-_kazemakase":
%%             titlepage for Graz University of Technology
%%             (correct new Corporate Design) by kazemakase (2013):
%%             see https://github.com/novoid/LaTeX-KOMA-template/issues/5
%% "template/title_VWA": titlepage for Vorwissenschaftliche Arbeit

\newcommand{\mytodonotesoptions}{}
%% e.g., "" (empty), "disable", ...
%% Options for the todonotes-package. If "disable", all todonotes will
%% be hidden (including listoftodos).

%% Load main settings for document preamble:
\input{template/preamble}%% DO NOT REMOVE THIS LINE!

\setboolean{myaddcolophon}{true}  %% "true" or "false"
%% If set to "true": a colophon (with notes about this document
%% template, LaTeX, ...) is added after the title page.
%% Please do not set to "false" without a good reason. The colophon
%% helps your readers to get in touch with LaTeX and to find this template.

\setboolean{myaddlistoftodos}{false}  %% "true" or "false"
%% If set to "true": the current list of open todos is added after the
%% table of contents. If \mytodonotesoptions is set to "disable", no
%% list of todos is added, independent of this setting here.

\setboolean{english_affidavit}{true}  %% "true" or "false"
%% If set to "true": the language of the statutory declaration text is set to
%% English, otherwise it is in German.


%% ========================================================================
%%%% Document metadata
%% ========================================================================

%% general metadata:
\newcommand{\myauthor}{AUTHOR}  %% also used for PDF metadata (hyperref)
\newcommand{\myauthorwithexistingtitles}{\myauthor{}, OLDDEGREE}  %% including
                                %% university degree already held
                                %% (BSc, MSc, ...)
\newcommand{\mytitle}{TITLE}  %% also used for PDF metadata (hyperref)
\newcommand{\mysubtitle}{ }  %% only used with title_Thesis_TU_Graz_-_kazemakase
\newcommand{\mysubject}{SUBJECT}  %% also used for PDF metadata (hyperref)
\newcommand{\mykeywords}{KEYWORDS}  %% also used for PDF metadata (hyperref)

%% this information is used only for generating the title page:
\newcommand{\myworktitle}{Master's Thesis}  %% official type of work like ``Master theses''
\newcommand{\mygrade}{Master of Science} %% title you are getting with this work like ``Master of ...''
\newcommand{\mystudy}{Telematik} %% your study like ``Arts''
\newcommand{\mydegreeprogramme}{Master's degree programme: \mystudy} %% Master's or PhD degree programme
\newcommand{\myuniversity}{Graz University of Technology} %% your university/school
\newcommand{\myfaculty}{ }  %% only used with title_Thesis_TU_Graz_-_kazemakase
\newcommand{\myinstitute}{Institute for Softwaretechnology} %% affiliation
\newcommand{\myinstitutehead}{Univ.-Prof.\,Dipl-Ing.\,Dr.techn.~Some One} %% head of institute
\newcommand{\mysupervisor}{Dr.~Some Body} %% your supervisor
\newcommand{\mycosupervisor}{\ }  %% only used with title_Thesis_TU_Graz_-_kazemakase
\newcommand{\myevaluator}{Prof.~Some Genius} %% your evaluator
\newcommand{\myhomestreet}{Street~42} %% your home street (with house number)
\newcommand{\myhometown}{Graz} %% your home town
\newcommand{\myhomepostalnumber}{8010} %% your postal number of home town
\newcommand{\mysubmissionmonth}{November} %% month you are handing in
\newcommand{\mysubmissionyear}{2013} %% year you are handing in
\newcommand{\mysubmissiontown}{\myhometown} %% town of handing in (or \myhometown)


%% additional information for generic_documentation title page
\newcommand{\myid}{1234567} %% Matrikelnummer
\newcommand{\mylecture}{LECTURE} %%


%% ========================================================================
%%%% MISC command definitions
%% ========================================================================
\input{template/mycommands}

%% ========================================================================
%%%% Typographic settings
%% ========================================================================
\input{template/typographic_settings}


%% ========================================================================
%%%% MISC usepackages
%% ========================================================================

%% ... it's OK to put here your own usepackage commands ...




%% ========================================================================
%%%% MISC self-defined commands and settings
%% ========================================================================

%% ... it's OK to put here your own newcommand/newenvironment-definitions ...




\newcommand{\myLaT}{\LaTeX{}@TUG\xspace} %% LaTeX@TUG text "logo"

\hyphenation{ex-am-ple hy-phen-ate}  %% in order to use German umlauts
%% here (Ver-\"of-fent-li-chung), you have to check for
%% activated \usepackage[T1]{fontenc} in the preamble

%% override default language of babel: (be sure to know, what you're
%% doing here)
%\selectlanguage{american}
%\selectlanguage{ngerman}

%% ========================================================================
%%%% Templates
%% ========================================================================

%% template for inserting figures:
% \myfig{}%% filename
%       {}%% width/height
%       {}%% caption
%       {}%% optional (short) caption for list of figures
%       {fig:}%% label

%% acronyms in small caps: \myacro{UNESCO}


\input{template/pdf_settings}  %% should be *last* definitions in preamble!
%% ========================================================================
%%%% begin{document}
%% ========================================================================
\begin{document}

\frontmatter                    %% KOMA: roman page numbers and such; only available in scrbook

\input{colophon}                %% defines information about editor, LaTeX, font, ...

%% Choose your desired title page:
\input{\mytitlepage}            %% include title page


\input{template/declaration_TU_Graz}  %% Statutory Declaration
% \input{thanks}                %% this is a suggestion: you have to create this file on demand
% \input{foreword}              %% this is a suggestion: you have to create this file on demand


%% include the abstract without chapter number but include it on table of contents:
\cleardoublepage
\phantomsection
\addcontentsline{toc}{chapter}{Abstract}
\include{abstract}              %% Abstract


\tableofcontents                %% this produces the table of contents - you might have guessed :-)

\listoffigures

%% if myaddlistoftodos is set to "true", the current list of open todos is added:
\ifthenelse{\boolean{myaddlistoftodos}}{
  \newpage\listoftodos          %% handy if you are using todonotes with \todo{}
}{}                             %% with todonotes-package option "disable" you can get rid of any todo in the output

\mainmatter                     %% KOMA: marks main part using arabic page numbers and such; only available in scrbook


%\input{example-short-chapter}   %% remove this line to get rid of the example chapter
%\input{example-style-chapter}   %% remove this line to get rid of the style chapter

%% include tex file chapters:
%----------------------------------------------------------------
%
%  File    :  thesis-style.tex
%
%  Author  :  Keith Andrews, IICM, TU Graz, Austria
% 
%  Created :  27 May 93
% 
%  Changed :  19 Feb 2004
% 
% styling and technical implementation adopted 2011 by Karl Voit
%----------------------------------------------------------------

%% defined an anvironment for the style Keith used to use:
\newenvironment{mykeithtabbing}[1]{%%
\begin{tabular}{lp{0.9\hsize}}
}{%%
\end{tabular}
}

\newcommand{\mybadgood}[2]{%%
\begin{mykeithtabbing}
{}\emph{Bad:}  & \sout{#1}  \\
\emph{Good:}   & #2  \\
\end{mykeithtabbing}

}

\chapter{Introduction}
\label{chap:Style}
\section{Motivation and Problem Statement}
\section{Thesis Contribution}
\section{Thesis Outline}


        %% this is a suggestion: you have to create this file on demand
%----------------------------------------------------------------
%
%  File    :  thesis-style.tex
%
%  Author  :  Keith Andrews, IICM, TU Graz, Austria
% 
%  Created :  27 May 93
% 
%  Changed :  19 Feb 2004
% 
% styling and technical implementation adopted 2011 by Karl Voit
%----------------------------------------------------------------

%% defined an anvironment for the style Keith used to use:


\chapter{Related Works}
\label{chap:Style}
\section{Drivable Area Detection}
\section{Ground Segmentation Techniques}
\section{Image and Object Segmentation}
\section{Real-Time Segmentation Networks}


             %% this is a suggestion: you have to create this file on demand
%----------------------------------------------------------------
%
%  File    :  thesis-style.tex
%
%  Author  :  Keith Andrews, IICM, TU Graz, Austria
% 
%  Created :  27 May 93
% 
%  Changed :  19 Feb 2004
% 
% styling and technical implementation adopted 2011 by Karl Voit
%----------------------------------------------------------------

%% defined an anvironment for the style Keith used to use:


\chapter{Methodology and Implementations}
\label{chap:Style}
\section{Datasets}
\section{Pseudo Ground Truth Generation Pipeline}
\section{Network Architecture}
\section{Implementation}


            %% this is a suggestion: you have to create this file on demand
%----------------------------------------------------------------
%
%  File    :  thesis-style.tex
%
%  Author  :  Keith Andrews, IICM, TU Graz, Austria
% 
%  Created :  27 May 93
% 
%  Changed :  19 Feb 2004
% 
% styling and technical implementation adopted 2011 by Karl Voit
%----------------------------------------------------------------

%% defined an anvironment for the style Keith used to use:


\chapter{Evaluations and Analysis}
\label{chap:Style}
\section{Evaluation Metrics}
\section{Ground Truth Pipeline and Network Benchmarking}
\section{Result Comparison}
\section{Quantitative Results}
\section{Ablation Study}

          %% this is a suggestion: you have to create this file on demand
%----------------------------------------------------------------
%
%  File    :  thesis-style.tex
%
%  Author  :  Keith Andrews, IICM, TU Graz, Austria
% 
%  Created :  27 May 93
% 
%  Changed :  19 Feb 2004
% 
% styling and technical implementation adopted 2011 by Karl Voit
%----------------------------------------------------------------

%% defined an anvironment for the style Keith used to use:


\chapter{Conclusion and Future Work}
\label{chap:Style}
\section{General Conclusion}
\section{Discussion and Future Work}

             %% this is a suggestion: you have to create this file on demand

\appendix                       %% closes main document, appendix follows until end; only available in book-classes
\addpart*{Appendix}             %% adding Appendix to tableofcontents

\printbibliography              %% remove, if using BibTeX instead of biblatex
% \include{further_ressources}  %% this is a suggestion: you have to create this file on demand






%%%% end{document}
\end{document}
%% vim:foldmethod=expr
%% vim:fde=getline(v\:lnum)=~'^%%%%\ .\\+'?'>1'\:'='
%%% Local Variables:
%%% mode: latex
%%% mode: auto-fill
%%% mode: flyspell
%%% eval: (ispell-change-dictionary "en_US")
%%% TeX-master: "main"
%%% End:
